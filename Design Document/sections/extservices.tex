The "Cracker" application uses Marvel Developer API in order to provide the user all the services. The SwiftyJSON library was used to parse the JSON data.

\vspace{8mm}

\subsection{Facebook SDK}
The Facebook SDK was used to manage the login through Facebook account. When a user wants to authenticate, the application makes the call to the Facebook service through:
\begin{lstlisting}
func loginButton(_ loginButton: FBLoginButton, didCompleteWith result: LoginManagerLoginResult?, error: Error?) {
	if let error = error {
		print(error.localizedDescription)
		return
	} else {
		let credential = FacebookAuthProvider.credential(withAccessToken: AccessToken.current!.tokenString)
		// then sign in through firebase with this credentials
	} 
}
\end{lstlisting}

This method is used to send the user on Facebook for the authentication; once authenticated, the user redirected to ”Cracker” which will store the token provided into Firebase.
The following pods were included to use Facebook:

\begin{lstlisting}
pod 'FacebookCore'
pod 'FacebookLogin'
\end{lstlisting}

\clearpage
  
  
\subsection{Google SDK}
The Google SDK was used to manage the login through Google account. When a user wants to authenticate, the application makes the call to the Google service through:
\begin{lstlisting}
func sign(_ signIn: GIDSignIn!, didSignInFor user: GIDGoogleUser!, withError error: Error!) {
	if let error = error {
		if (error as NSError).code == GIDSignInErrorCode.hasNoAuthInKeychain.rawValue {
			print("The user has not signed in before or they have since signed out.")
		} else {
			print("\(error.localizedDescription)")
		}
		return
	} else {
		guard let authentication = user.authentication else { return }
		let credential = GoogleAuthProvider.credential(withIDToken: authentication.idToken, accessToken: authentication.accessToken)
		// then sign in through firebase with this credentials
	}
}
\end{lstlisting}
This method is used to send the user on Google for the authentication; once authenticated, the user redirected to ”Cracker” which will store the token provided into Firebase.
The following pods were included to use Google:
\begin{lstlisting}
pod 'GoogleSignIn'
\end{lstlisting}
    
\clearpage
    
    
\subsection{Firebase SDK}
Firebase SDKs are used to access the database, to login with email and password and to save other authentication through Facebook and Google.
\begin{lstlisting}
Auth.auth().signIn(with: credential) { (authResult, error) in
	if error == error{
		print("Error Log In Firebase \(error)")
		return
	} else {
		print("Logged In Firebase")
	}
}
\end{lstlisting}

For database access, the components included in the SDKs was used:
\begin{lstlisting}
let user = Firestore.firestore().collection("Users").document("\((Auth.auth().currentUser?.uid)!)")
User.collection("Series").getDocuments() { (querySnapshot, err) in
	if let err = err {
		print("Error getting documents: \(err)")
	} else {
		print("Collection got")
		...
	}
}
\end{lstlisting}

The following pods were included to use Firebase:
\begin{lstlisting}
pod 'Firebase/Auth'
pod 'Firebase/Firestore'
pod 'Firebase/Analytics'
pod 'FirebaseFirestoreSwift'
\end{lstlisting}


\clearpage

\subsection{Alamofire}
Alamofire is used to send GET requests to the endpoint supplied by Marvel APIs. If the answer is valid, it returns a JSON file that can be parsed.
\begin{lstlisting}
Alamofire.request(url, method: .get, parameters: parameters).responseJSON { response in
	if response.result.isSuccess {
		print("Success! Got the data")
		let json : JSON = JSON(response.result.value!)
	} else {
		print("Error \(String(describing: response.result.error))")
	}
}
\end{lstlisting}
The following pods were included to use Alamofire:
\begin{lstlisting}
pod 'Alamofire'
\end{lstlisting}

\clearpage

\subsection{Marvel Comics API}
The Marvel Comics API is a RESTful service which provides methods for accessing spe- cific resources at canonical URLs and for searching and filtering sets of resources by various criteria. All representations are encoded as JSON objects. All documentation can be found here: https://developer.marvel.com/

\subsubsection{Service Endpoint}
The Marvel Comics API’s base endpoint is http(s)://gateway.marvel.com/

\subsubsection{Resources}
You can access six resource types using the API:
\begin{itemize}
\item {\textbf{Comics}}: individual print and digital comic issues, collections and graphic novels.
\item {\textbf{Comics series}}: sequentially numbered (well, mostly sequentially numbered) groups comics with the same title.
\item {\textbf{Comics stories}}: indivisible, reusable components of comics. For example, the cover from Amazing Fantasy 15 or the origin of Spider-Man story from that comic.
\item {\textbf{Comics events and crossover}}: big, universe-altering storylines.
\item {\textbf{Creators}}: women, men and organizations who create comics.
\item {\textbf{Characters}}: the women, men, organizations, alien species, deities, animals, non- corporeal entities, trans-dimensional manifestations, abstract personifications, and green amorphous blobs which occupy the Marvel Universe (and various alternate universes, timelines and altered realities therein).
\end{itemize}

Results returned by the API endpoints have the same general format, no matter which entity type the endpoint returns. Every successful call will return a wrapper object, which contains metadata about the call and a container object, which displays pagination information and an array of the results returned by this call. This pattern is consistent even if you are requesting a single object.


Putting it together, a typical result will look like this:
\begin{lstlisting}
{
	"code": 200,
	"status": "Ok",
	"etag": "f0fbae65eb2f8f28bdeea0a29be8749a4e67acb3",
	"data": {
		"offset": 0,
		"limit": 20,
		"total": 30920,
		"count": 20,
		"results": [{array of objects}}]
	}
}
\end{lstlisting}


\subsubsection{Authentication}
All requests to the APIs must be authenticated using the methods outlined in the request signing and authentication guidelines. Requests which fail authentication generally pass a 401 HTTP code and an error describing the reason for rejection.

























