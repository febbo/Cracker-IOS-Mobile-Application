\subsection{Overview}
In this section we describe the architectural design of our system, the main components and their interactions. \\
The system will be described starting with high-level components. Then a more detailed description is provided for the client and the database structure, including the architectural patterns applied.

\subsection{High level components}
The main high level components of our system are:
\begin{itemize}
\item {\textbf{Client}}: this component is responsible for the visualization of data (front-end) and for the management of requests and replies to and from the services providing data to the system.
\item {\textbf{Firebase}}: this component is responsible for the storage of the permanent data of the users and for the registration and login of users; we use the services provided by Firebase to manage these data.
\item {\textbf{Marvel APIs}}: we use the APIs provided by Marvel in order to always have complete and up-to-date data on comics.
\item {\textbf{Facebook}}: used for the login procedure.
\item {\textbf{Google}}: used for the login procedure.
\end{itemize}

\vspace{3mm}

\begin{figure}[h]
\centering
\includegraphics[width=\textwidth]{img/components}
\caption{High Level Component View}
\end{figure}

\clearpage

\subsection{Client}
For the implementation of the application we have chosen a mobile back-end, that is a client architecture. This choice was made mainly because the application does not interface with other users and because for various services it uses third-party APIs. Communication with third-party services is based on HTTPS REST requests, in particular through GET requests.
The client uses the traditional MVC pattern:
\begin{itemize}
\item Model: this package contains all the classes representing data to be shown to the single user, taken by the Controller and published by the View.
\item View: this package contains all the components that display data to the user and interact with him.
\item Controller: this package contains all the objects in charge to interact between one or more view objects of the application and one or more model objects.
\end{itemize}

\subsection{Database}
Since the application receives all the necessary data from external services through the API, the only data that need to be saved are the informations about the user, the series that fol- lows and the comics read. Moreover, since there is no interaction between the different users, this data are saved on the Cloud Firestore on Firebase Platform to guarantee ACID properties. The database interacts only with the application layer. The security restriction will be implemented to guarantee the user privacy from unauthorized users. The communi- cation between the database and application tier has to be encrypted. The access to the data has to be guaranteed only to the authorized user for that data.

\begin{figure}[h]
\centering
\includegraphics[width=\textwidth]{img/DB}
\caption{Database Cloud FIrestore}
\end{figure}

\clearpage

The E / R model of the data collected through the API together with the user data saved on the firebase is as follows:


\begin{figure}[h]
\centering
\includegraphics[width=\textwidth]{img/ER}
\caption{E/R Diagram}
\end{figure}


